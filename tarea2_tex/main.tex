\documentclass[conference]{IEEEtran}

\usepackage{graphicx}
\usepackage{subfigure}
\usepackage{amsmath} % assumes amsmath package installed
\usepackage{amssymb} 
\usepackage{array}
\usepackage{tikz,circuitikz}
\usetikzlibrary{fit}
\usepackage{blindtext}
\usepackage{color}
\usepackage{siunitx}
\usepackage{float} % enables [H] float placement

\def\BibTeX{{\rm B\kern-.05em{\sc i\kern-.025em b}\kern-.08em
    T\kern-.1667em\lower.7ex\hbox{E}\kern-.125emX}}


\usepackage{mwe}
\usepackage{fancyhdr}
\fancypagestyle{firststyle}
{
	\fancyhf[C]{\fontsize{8}{10} \selectfont \textit{} }
	\fancyfoot[C]{}
}


\hyphenation{op-tical net-works semi-conduc-tor}


\begin{document}
%
% paper title
% Titles are only capitalized in the first letter.
% Linebreaks \\ can be used within to get better formatting as desired.
% Do not put math or special symbols in the title.
\title{Análisis de un rectificador de media onda con carga RL}

\author{
\IEEEauthorblockN{Yosniel Agüero}
\IEEEauthorblockA{\textit{Universidad de Guadalajara}\\
MCIE\\
Guadalajara, México\\
\scriptsize{\textit{yosniel.aguero9368@alumnos.udg.mx}}}
\and
\IEEEauthorblockN{Glader Hernandez}
\IEEEauthorblockA{\textit{Universidad de Guadalajara}\\
MCIE\\
Guadalajara, México\\
\scriptsize{\textit{glader.hernandez9367@alumnos.udg.mx}}}
\and
\IEEEauthorblockN{Gary Sosa}
\IEEEauthorblockA{\textit{Universidad de Guadalajara}\\
MCIE\\
Guadalajara, México\\
\scriptsize{\textit{gary.sosa9369@alumnos.udg.mx}}}
\and
\IEEEauthorblockN{Ulrik Wong}
\IEEEauthorblockA{\textit{Universidad de Guadalajara}\\
MCIE\\
Guadalajara, México\\
\scriptsize{\textit{ulrik.wong7998@alumnos.udg.mx}}}
}


\maketitle

\thispagestyle{firststyle}
\renewcommand{\headrulewidth}{0in}
\pagestyle{empty}


\pagestyle{fancy}
\chead{\fontsize{8}{10} \selectfont \textit{} }
\pagenumbering{gobble}



% As a general rule, do not put math, special symbols or citations
% in the abstract
\begin{abstract}
This document is a model and instructions for \LaTeX.
This and the IEEEtran.cls file define the components of your document [title, text, heads, etc.]. *CRITICAL: Do Not Use Symbols, Special Characters, Footnotes, 
or Math in Document title or Abstract.
\end{abstract}


\IEEEpeerreviewmaketitle



\section{Introduction}
This document is a model and instructions for \LaTeX.
Please observe the report page limits.
 
\begin{figure}[ht]
	\centering
	\begin{circuitikz}[american, cute inductors, american currents, american voltages]
		\ctikzset{diodes/scale=0.5, sources/scale=0.75, resistors/scale=0.5, inductors/scale=0.75}
		\draw (0,0) to[sinusoidal voltage source, v^<=$v_i$] (0,2) to [full diode, l_=$D_1$, v^=$v_{D}$, i_=$i_D$](2,2);
		\draw (2,0) to[full diode, l^=$D_2$, v_<=$v_{oi}$](2,2) to[L, l_=$L$, i_=$i_L$, v^=$v_L$] (4,2);
		\draw (4,2)-- (5,2) to[R, l_=$R$, v^=$v_R$, i>_=$i_R$] (5,0);
		\draw (0,0) -- (5,0);
	\end{circuitikz}
\end{figure}
\section{Análisis del rectificador monofásico de media onda con carga \(R\!-\!L\) y diodo de corrida libre}
En lo que sigue consideramos la fuente sinusoidal
\[
v_i(t)=V_m\sin(\omega t),
\]
una impedancia serie \(R\) y \(L\) (carga), y dos diodos ideales:
\begin{itemize}
  \item \(D_1\): rectificador que conecta la fuente a la carga (ánodo en la fuente, cátodo en el nodo de carga).
  \item \(D_2\): diodo de \emph{flyback } en paralelo a la carga (orientación tal que permite mantener la corriente inductiva cuando la fuente deja de alimentar).
\end{itemize}

\subsection{Estados asumidos de los diodos}

Definición de los 4 estados considerados y denotamos ON como  conducción (diodo polarizado directamente) y OFF como bloqueo (polarizado inversamente).
\[
\begin{array}{ll}
\text{Estado A: } & (D_1=\mathrm{ON},\; D_2=\mathrm{OFF}) \\
\text{Estado B: } & (D_1=\mathrm{OFF},\; D_2=\mathrm{ON}) \\
\text{Estado C: } & (D_1=\mathrm{OFF},\; D_2=\mathrm{OFF}) \\
\text{Estado D: } & (D_1=\mathrm{ON},\; D_2=\mathrm{ON}) \\
\end{array}
\]


\begin{itemize}
  \item \textbf{Operación de cada estado asumido :}
    \begin{enumerate}
      \item Estado A: Ocurre cuando la tensión instantánea de la fuente tiende a \emph{polarizar positivamente} \(D_1\) y puede imponer una tensión mayor en el nodo de carga que la necesaria para forzar \(D_2\) en conducción inversa. \(D_1\) conduce si \(\;v_i(t) > v_R(t)\) (ánodo de \(D_1\) más positivo que su cátodo). En la práctica, con diodos ideales y caída nula, el encendido ocurre cuando \(v_i(t)\) supera la tensión instantánea necesaria para mantener la corriente \(i(t)>0\).
      \item Estado B: Ocurre cuando la fuente no sostiene la corriente inductiva, pero la inercia del inductor mantiene corriente positiva; entonces \(D_2\) ofrece el camino de libre. \(D_2\) conduce si la polaridad en la carga, hace que el ánodo de \(D_2\) sea más positivo que su cátodo, es decir, cuando la inercia del inductor empuja la corriente y la tensión en la carga favorece la conducción por \(D_2\). Con la orientación antiparalela típica, esto sucede cuando \(v_i(t)\) cae por debajo de \(v_o(t)\) y la corriente inductiva es positiva.
      \item Estado C: Estado de no-conducción es válido si la corriente ha decaído a cero (\(i(t)=0\)  ) y ambas tensiones en los ánodos/cátodos de los diodos dejan a ambos inversamente polarizados.
      \item Estado D: Si los diodos son ideales y están orientados en la configuración habitual (antiparalelo para \(D_2\)), la conducción simultánea tiende a producir una contradicción en las polaridades o a generar un lazo de circulación que, en el mejor de los casos, fuerza igualar tensiones entre nodos y, en el peor, crea una trayectoria de cortocircuito entre la fuente y el retorno. Por tanto en la práctica se considera no-sostenible como estado permanente y debe justificarse con cuidado.
\end{enumerate}
\end{itemize}

\subsection{Análisis de estado estacionario periódico}


\subsection{Modelo en espacio de estados de los estados válidos }
En este sistema la única variable de estado es la corriente \(i(t)\) de la carga \(R\!-\!L\). Usaremos la variable de estado:
\[
x(t)=i(t).
\]
A continuación planteamos las ecuaciones de estado para cada configuración válida (y para la configuración asumida D).

\paragraph{Estado A: \(D_1=\mathrm{ON},\;D_2=\mathrm{OFF}\) (circuito L–R  con fuente de alimentación).}
La ecuación diferencial y la representación en espacio de estados:
\[
L\frac{di}{dt} + R\,i(t) = v_s(t) = V_m\sin(\omega t).
\]
\[
\begin{bmatrix}\dot{x}(t)\end{bmatrix}
=
\begin{bmatrix}-\dfrac{R}{L}\end{bmatrix}
\begin{bmatrix}x(t)\end{bmatrix}
+
\begin{bmatrix}\dfrac{1}{L}\end{bmatrix}
\qquad
y(t)=\begin{bmatrix}R\end{bmatrix}\begin{bmatrix}x(t)\end{bmatrix}.
\]
Solución general (en \(t_0\) con condición \(i(t_0)=I_0\)):
\[
i(t)=e^{-\frac{R}{L}(t-t_0)}I_0 + \frac{V_m}{L}\int_{t_0}^{t} e^{-\frac{R}{L}(t-\tau)}\sin(\omega \tau)\,d\tau.
\]
La parte forzada en régimen permanente (cuando el tiempo es mucho mayor que \(\tfrac{L}{R}\)) tiene la forma:
\[
i_p(t)=\frac{V_m}{\sqrt{R^2+(\omega L)^2}}\sin(\omega t - \phi),\qquad \phi=\arctan\!\frac{\omega L}{R}.
\]

\paragraph{Estado B: (diodo \emph{flyback } — circuito L–R  sin fuente).}
Ecuación diferencial:
\[
L\frac{di}{dt} + R\,i(t) = 0
\]
\[
\begin{bmatrix}\dot{x}(t)\end{bmatrix}
=
\begin{bmatrix}-\dfrac{R}{L}\end{bmatrix}
\begin{bmatrix}x(t)\end{bmatrix}
\qquad
y(t)=\begin{bmatrix}R\end{bmatrix}\begin{bmatrix}x(t)\end{bmatrix}.
\]
La solucion general es:
\[
i(t)=I_1\,e^{-\frac{R}{L}(t-t_1)},
\]
donde \(I_1\) es la corriente en el instante \(t_1\) en que comienza el freewheeling.
\paragraph{Estado C: (no-conducción).}
Si \(x(t)=i(t)=0\) este estado es consistente; la dinámica es \(i\equiv 0\). Si el inductor intenta imponer \(i\neq0\). 



\begin{thebibliography}{plain}
\bibitem{b1} G. Eason, B. Noble, and I. N. Sneddon, ``On certain integrals of Lipschitz-Hankel type involving products of Bessel functions,'' Phil. Trans. Roy. Soc. London, vol. A247, pp. 529--551, April 1955.
\bibitem{b2} J. Clerk Maxwell, A Treatise on Electricity and Magnetism, 3rd ed., vol. 2. Oxford: Clarendon, 1892, pp.68--73.
\bibitem{b3} I. S. Jacobs and C. P. Bean, ``Fine particles, thin films and exchange anisotropy,'' in Magnetism, vol. III, G. T. Rado and H. Suhl, Eds. New York: Academic, 1963, pp. 271--350.
\bibitem{b4} K. Elissa, ``Title of paper if known,'' unpublished.
\bibitem{b5} R. Nicole, ``Title of paper with only first word capitalized,'' J. Name Stand. Abbrev., in press.
\bibitem{b6} Y. Yorozu, M. Hirano, K. Oka, and Y. Tagawa, ``Electron spectroscopy studies on magneto-optical media and plastic substrate interf	ace,'' IEEE Transl. J. Magn. Japan, vol. 2, pp. 740--741, August 1987 [Digests 9th Annual Conf. Magnetics Japan, p. 301, 1982].
\bibitem{b7} M. Young, The Technical Writer's Handbook. Mill Valley, CA: University Science, 1989.
\end{thebibliography}

\vspace{12pt}
\color{red}
This report template contains guidance text for composing and formatting documents. Please ensure that all template text is removed from your document prior to submission of your work. Failure to remove the template text from your document will be considered as a lack of editing effort.

\end{document}